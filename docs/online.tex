\section{Online Learner}
One algorithm is straightforward: for meeting belief aggregation, ensure that after each agent has participated, the market price reflects the aggregate belief of all the agents that have participated so far.

\subsection{Background}
Since $\LMSRb$ controls how quickly agents can move the market price, a minuscule $\LMSRb$ value permits an agent to set the market price to their belief at almost no cost while a large value of $\LMSRb$ makes moving the market price to their belief almost infinitely expensive. Agents with unit budgets, per our model, can move the market price to their belief without exhausting their budgets when $\LMSRb \leq 1$. Similarly, agents with unit budgets cannot move the price more than a rounding error when $\LMSRb \ge 100$. Since the LMSR cost function is both continuous and differentiable on that interval, the Mean Value Theorem implies that $\LMSRb$ can be set to recover any value on the interval $[\OptionPrice_t, \Agent_\AgentBelief]$. Agents will always exhaust their budgets moving the market price as close as possible to their beliefs.\\

When operating on the real line, the definition of an arithmetic mean of arbitrary values $a$ and $b$ constrains the result to the interval $[a,b]$. We can utilize this fact to ensure that the market price is always equal to the arithmetic mean of the agent beliefs that the market maker has seen so far. Formally: $\OptionPrice_t = \frac{\sum_{i=1}^{t} \Agent^{i}_\AgentBelief}{t}$, $\forall t \in \{1,|\Agents|\}$.\\

\subsection{Algorithm}
The Online Learner operates by selecting a minuscule $\LMSRb$ for the first agent that arrives. After the first agent has participated, the price will be equal to first agent's belief. For all periods afterwards, the algorithm will receive the new agent's belief, update its aggregate belief by computing the new average accounting for that agent, and solve for $\LMSRb$ such that the new agent will exhaust its budget by moving the market price to the new aggregate belief. A trivial way to solve for $\LMSRb$ is by conducting a binary search on the interval $(0, 100]$. If multiple $\LMSRb$ values solve the system then the Online Learner selects the smallest in order to minimize cost.\\

The following is an implementation of the Online Learner that for clarity does not use binary search:\\
\begin{algorithm}[H]
\SetAlgoLined
\TitleOfAlgo{setB}
\SetKwInOut{Input}{input}
\SetKwInOut{Output}{output}
\Input{state $\left<\QuantityYes, \QuantityNo\right>$, price $p$, new belief $\Agent_\AgentBelief$, old average $a$, time $\Time$}
\Output{liquidity parameter $\LMSRb$}
newAverage = $\frac{a\Time + \Agent_\AgentBelief}{\Time + 1}$\\
newB = $\epsilon$\\
\While{true}{
	direction = newBelief $> p$\\
	desiredShareNumber = quantity($\left<\QuantityYes, \QuantityNo\right>$, newB, $\Agent_\AgentBelief$, direction)\\
	\eIf{direction}{
		desiredCost = cost(desiredShareNumber, 0)
	}{
		desiredCost = cost(0, desiredShareNumber)
	}

	\If{desiredCost $> \Agent_\AgentBudget$}{
		desiredShareNumber = capitalToShares($\left<\QuantityYes, \QuantityNo\right>$, newB, $\Agent_\AgentBudget$, direction)
	}

	newState = $\left<\QuantityYes, \QuantityNo\right> + $ desiredShareNumber\\
	newPrice = price(newState, newB)\\
	\eIf{newAverage $\neq$ newPrice}{
		newB = newB + $\epsilon$
	}{
		return newB
	}
}
\end{algorithm}