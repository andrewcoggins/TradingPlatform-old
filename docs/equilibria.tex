\section{Rational Expectations Equilibrium}
In order to constrain our algorithm for setting the liquidity parameter $\LMSRb$ such that the final market price is equal to the aggregate belief, we first must define the aggregate belief. We will use a standard solution concept from the literature.\\

<<<<<<< HEAD
\subsection{RE and PI}
The two classic equilibria concepts that we will utilize are the \mydef{Rational Expectations Equilibirium} (RE) and the \mydef{Prior Information Equilibrium} (PI). Rational Expectations hypothesizes that all agents act as if they had the collective signal. The collective signal is the signal that aggregates the individual signals received by each agent. This implies that the prediction markets should be as accurate as the collective signal. Under our model, this implies that the prediction markets' final prediction should predict the true outcome. \\

\mydef{Prior Information} hypothesizes that all agents act on a linear combination of their private information and the current market price as Bayesian updaters. PI implies that agents will take into account the current market price and their own signal. Agents are willing to participate at their expected value given the two signals. \\

A major flaw in the current literature, which is reflected in LMSR and the agent behavior theories, is the generalization that agents cannot choose when to enter the market. Although fixing entry time is useful for equilibrium calculations nevertheless it is important to theorize about the value of a certain entry time $\Time$. Under a model based in RE where the collective signal is truthful, a rational agent would want to be the last decision maker and then have the outcome revealed with certainty.\\

\subsection{No Trade Theorem}
Under relaxed assumptions from RE and PI, agents wait until $x \in \left[0, |\Agents| \right]$ agents have executed
their strategies depending on their valuation of the information gained by waiting. It is trivial to show that 
the valuations are monotonically increasing with $x$. When $x \geq HOLD$, agents can 
determine with certainty what the outcome will be, incur no risk, and therefore $\forall x \geq HOLD = E\left[\Option\right]$.
Each spot  $x \leq HOLD$ is worth $\frac{x}{|\Agents|} 
E\left[\Option\right]$. This implies that rational agents
would need to be paid the difference between their riskless profits and the value of $x$ in order to select spot $x < HOLD$: $E\left[\Option\right](1 - \frac{x}{|\Agents|})$. In the absence of this payment, it is individually rational for agents to wait until the first half of agents have already entered the market before trading. Therefore, lacking any external payments, no agent will enter the market and no
trades will occur.

\subsection{Prior Information Timing}
We present a novel theory called \mydef{Prior Information Timing} (PIT) that takes into account our No Trade
Theorem where agents choose valuations based on a linear combination of their signal and the market
price, but value deferring this assesment in order to gain more information. Risk neutral agents can
price the ability to defer the decision until position $x$ based on the added information of the 
preceding $x-1$ agents can pay up to the value of the information to defer. In order for markets
to clear in PIT then the market maker needs to compensate agents for their spot selection.\\

\subsection{Examples}
\mydef{Rational Expectations} implies that the final price will be the average of each agent's signal since all agents have equal budgets. As we have restricted signals to the set of ${0,1}$, the expected price will simply be the number of 1 signals divided by the total number of agents $\frac{|\Agents_{\Option_1}|}{|\Agents_\Option|}$.  For example, with a signal vector $\left[0,1,1,0,1\right]$, the expected price is .6 and the prediction will be correct. \\

\mydef{Prior Information} implies that the final price will be between the weighted average and .5 because it is biased towards the initial price. The prediction will still be accurate just with lower confidence. For example, with a signal vector $\left[0,1,1\right]$, the first agent will target .25 since it is the Bayesian update between their prior 0 and the price .5. The second agent will target .5 since it is the Bayesian update between their prior 1 and the price .25 giving .25 double weight. The third agent will target .625 since it is the Bayesian update between their prior 1 and the price .5 giving .5 triple weight. The outcome will still be correct at .625, but less than .66 which is the RE prediction in this case.\\

\mydef{Prior Information Timing} implies something that I will calculate.\\

\subsection{Agent Strategy}
An agent's $\Agent\in\Agents_\Option$ \mydef{strategy} 
$\AgentStrategy_\Agent(\Time) \in \mathbb{R}_+$ specifies the quantity of option 
$\Option$ purchased by the agent at time $\Time$.\\

In order to assess accuracy in situations where agents have variable information, we will consider different types of agent strategies in our experiments. There are two types of agent strategies \mydef{myopic} $\Myopic$ and \mydef{farsighted} $\Farsighted$. Prediction markets are known to be myopically incentive compatible, which means that myopic agents bid truthfully. Similarly, \mydef{informed} $\Informed$ agents have an exogenous signal about the outcome whereas \mydef{uninformed} $\Uninformed$ agents can only base their decision on the market price. \\

The following agent strategies build off of the work on Kelly Agents, Constant Relative Risk Aversion Agents, (Kets et al, 2014) and Zero Intelligence Agents (Othman, 2008). Kelly and CRRA Agents are limiting because they introduce risk aversion in order to limit the bets that agents place. This is empirically less accurate since online betting environments are known to attract risk seeking, or at least risk neutral, traders. Zero Intelligence Agents inspire our Myopic class of agents since they are short sighted actors. Kelly and CRRA Agents inspire our Uninformed class of agents since their purpose is to measure the persistence of inaccurate traders. We believe that \mydef{Myopic, Farsighted, and Un/Informed} agents cover a more abstract class of trader behavior.\\

$\AgentStrategy^{\Myopic\Informed}_\Agent(\Time)$ is an \mydef{informed agent} holding exogenous signal $\Agent_\AgentBelief$ who is willing to pay up to $\Agent_\AgentBudget$ in order to move the market price $\OptionPrice_\Market$ as close as possible to their belief $\Agent_\AgentBelief$. \\

$\AgentStrategy^{\Myopic\Uninformed}_\Agent(\Time)$ is an \mydef{uninformed agent} holding no exogenous signal $\Agent_\AgentBelief$ who who is willing to pay up to $\Agent_\AgentBudget$ in order to move the market price $\OptionPrice_\Market$ as close as possible to 1 if at time $\Time$, $\OptionPrice_{\Market\Time} \ge .5$ otherwise 0. \\

$\AgentStrategy^{\Farsighted\Informed}_\Agent(\Time)$ is an \mydef{informed agent}holding exogenous signal $\Agent_\AgentBelief$ who is attempting to maximize the expected value by bidding based on a linear combination of their signal $\Agent_\AgentBelief$ and the current market price $\OptionPrice_\Market$ accounting for how many agents $x$ have already bid. \\

$\AgentStrategy^{\Farsighted\Uninformed}_\Agent(\Time)$ is an \mydef{uninformed agent} holding exogenous signal $\Agent_\AgentBelief$ who is attempting to maximize the expected value by bidding based on the current market price $\OptionPrice_\Market$ accounting for how many agents $x$ have already bid. \\
=======
The \mydef{Rational Expectations Equilibrium} (REE) hypothesizes that all agents act as if they knew the private belief $\Agent_\AgentBelief$ of all other agents \cite{10.2307/1911360}. This fully revealing hypothesis implies that all agents would operate as if their belief were the aggregate belief constructed from the pooled information. As a result, the equilibrium price should be the aggregate belief regardless of the market mechanism.\\

Since REE is a perfectly competitive equilibrium concept, the aggregate belief is the budget weighted average of each of the agents private beliefs. Formally, $\frac{\sum_{i=1}^{|\Agents|} (\Agent^{i}_{\AgentBelief} \Agent^{i}_{\AgentBudget})}{\sum_{j=1}^{|\Agents|} \Agent^{j}_{\AgentBudget}}$. When agents have equal budgets, the REE reduces to: $\frac{\sum_{i=1}^{|\Agents|} \Agent^{i}_{\AgentBelief}}{|\Agents|}$.\\
>>>>>>> newInterfaces
