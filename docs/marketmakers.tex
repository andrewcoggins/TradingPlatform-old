\section{\href{http://mason.gmu.edu/~rhanson/mktscore.pdf}{Logarithmic Market Scoring Rule}}
LMSR is a strictly proper scoring rule proposed by Robin Hanson \cite{Hanson2003}. A scoring rule charges agents depending on the revealed accuracy of their beliefs. A scoring rule is strictly proper if an agent has no incentive to report anything but their true belief. \\ 

LMSR is defined by a logarithmic cost function: 
$C(\QuantityYes,\QuantityNo) = \LMSRb\ln\left(e^{\frac{\QuantityYes}{\LMSRb}} + e^{\frac{\QuantityNo}{\LMSRb}}\right)$. The instantaneous price of market $\Market$, which is also the market's prediction for the option $\Option_{\{0,1\}}$, is quoted with: $p_{\{0,1\}} = \frac{e^{\frac{q_{\{0,1\}}}{\LMSRb}}}{e^{\frac{\QuantityYes}{\LMSRb}} + e^{\frac{\QuantityNo}{\LMSRb}}}$. The cost charged to a trader wanting to buy $q_a$ shares of $\Option_0$ and $q_b$ shares of $\Option_1$ is: $C(\QuantityYes + q_a, \QuantityNo + q_b) - C(q_1,\QuantityNo)$. Agents are charged to move the market prediction. The liquidity parameter $\LMSRb$ is set \emph{ex ante} by the market maker. It controls how much the market maker can lose and also adjusts how easily a trader can change the market price. The market maker never loses more than $\LMSRb\ln(2)$. A large $\LMSRb$ means that it would cost a lot to move the market price while a small $\LMSRb$ makes large swings relatively inexpensive.\\

Although $\LMSRb$ was designed to be a static parameter, recent papers discuss updating $\LMSRb$ as the state of the market changes \cite{Othman:2013:PLA:2509413.2509414}. By updating $\LMSRb$, the market maker can change how quickly the price fluctuates and how much loss they are willing to accept. The goal of this work is to find a novel approach to setting a sequence of $\LMSRb$'s that minimizes loss and recovers the aggregate belief.\\

\textbf{Advantages}\\
\begin{enumerate}
\item{Path Independence - any way the market moves from one state to another state yields the same payment or cost to the traders in aggregate \cite{Hanson2003}.}
\item{Translation Invariance - all prices sum to unity. This ensures that the price of each outcome state maps to a probability.}
\end{enumerate}

\textbf{Disadvantages} \\
\begin{enumerate}
\item{Liquidity Insensitive - for a fixed $\LMSRb$, the market cannot adjust to periods with low or high activity. The market maker must set the liquidity parameter based on their prior belief, but has little to no guidance on how to set it \cite{Othman:2013:PLA:2509413.2509414}.}
\item{Guaranteed Loss - the market maker cannot profit, but has a bounded loss.}
\end{enumerate}

\subsection{LMSR Algorithms}
\begin{algorithm}[H]
\SetAlgoLined
\TitleOfAlgo{cost}
\SetKwInOut{Input}{input}
\SetKwInOut{Output}{output}
\Input{state $ \left<\QuantityYes, \QuantityNo\right>$, liquidity parameter $\LMSRb$, to trade $\Quantity$, direction $\in \{\text{true},\text{false}\}$}
\Output{cost}
oldScore = $\LMSRb \left(\exp\left(\frac{\QuantityYes}{\LMSRb}\right) + \exp\left(\frac{\QuantityNo}{\LMSRb}\right)\right)$\\
\eIf{direction}{
	newScore = $\LMSRb \left(\exp\left(\frac{\QuantityYes + \Quantity}{\LMSRb}\right) + \exp\left(\frac{\QuantityNo}{\LMSRb}\right)\right)$\
}{
	newScore = $\LMSRb \left(\exp\left(\frac{\QuantityYes}{\LMSRb}\right) + \exp\left(\frac{\QuantityNo + \Quantity}{\LMSRb}\right)\right)$\
}

return newScore - oldScore\
\end{algorithm}

\begin{algorithm}[H]
\SetAlgoLined
\TitleOfAlgo{price}
\SetKwInOut{Input}{input}
\SetKwInOut{Output}{output}
\Input{state $\left<\QuantityYes, \QuantityNo\right>$, liquidity parameter $\LMSRb$}
\Output{price}
quantity, price = getQuantityPrice(state, $\LMSRb$, null, true)\\
return price\\
\end{algorithm}

\begin{algorithm}[H]
\SetAlgoLined
\TitleOfAlgo{quantity}
\SetKwInOut{Input}{input}
\SetKwInOut{Output}{output}
\Input{state $\left<\QuantityYes, \QuantityNo\right>$, liquidity parameter $\LMSRb$, belief $\Agent_\AgentBelief$, direction $\in \{\text{true},\text{false}\}$}
\Output{quantity $\Quantity$}
quantity, price = getQuantityPrice(state, $\LMSRb$, belief, direction)\\
return quantity\\
\end{algorithm}
\newpage

\begin{algorithm}[H]
\SetAlgoLined
\TitleOfAlgo{getQuantityPrice}
\SetKwInOut{Input}{input}
\SetKwInOut{Output}{output}
\Input{state $\left<\QuantityYes, \QuantityNo\right>$, liquidity parameter $\LMSRb$, belief $\Agent_\AgentBelief$, direction $\in \{\text{true},\text{false}\}$}
\Output{quantity $\Quantity$, price $\MarketPrice$}
\eIf{belief == null}{
	quantity = 0
}{
	\eIf{direction}{
		price = belief\\
		side = $\QuantityYes$\\
		top = $\QuantityNo$\
	}{
		price = 1 - belief\\
		side = $\QuantityNo$\\
		top = $\QuantityYes$\
	}
	quantity = $\LMSRb \log\left(\frac{\exp\left(\frac{\text{top}}{\LMSRb}\right)\text{price}}{1-\text{price}}\right) - \text{side}$\\
}
\eIf{direction}{
	newPrice = price($\QuantityYes$ + quantity, $\QuantityNo$, $\LMSRb$)
}{
	newPrice = price($\QuantityYes$, $\QuantityNo$ + quantity, $\LMSRb$)	
}
return (quantity, newPrice)\\
\end{algorithm}

\begin{algorithm}[H]
\TitleOfAlgo{capitalToShares}
\SetKwInOut{Input}{input}
\SetKwInOut{Output}{output}
\Input{market maker $\MarketMaker = \left<\QuantityYes, \QuantityNo, \LMSRb\right>$, money $\Money$, direction $\in \{\text{true},\text{false}\}$}
\Output{quantity}
\eIf{direction}{
side = $\QuantityYes$\
top = $\QuantityNo$\
}{
side = $\QuantityNo$\
top = $\QuantityYes$\
}
return $\LMSRb 
\log\left(
	\exp\left(
		\frac{\Money}{\LMSRb} + 
		\log\left(
			\exp\left(\frac{\Market_\QuantityYes}{\LMSRb}\right) + 
			\exp\left(\frac{\Market_\QuantityNo}{\LMSRb}\right)
		\right)
	\right) - 
	\exp\left(\frac{\text{top}}{\LMSRb}
	\right)
\right) - \text{side}$
\end{algorithm}

\subsection{Agent Strategy}
The Logarithmic Market Scoring rule is known to be myopically incentive compatible, which means that myopic agents bid truthfully. There is much work demonstrating that this assumption fails when agents can use \mydef{bluffing} and \mydef{reticence} to mislead other agents and profit off of that deception \cite{Chen2007}. \mydef{Bluffing} occurs when an agent moves the market price away from their belief, and \mydef{reticence} occurs when the agent moves the market price towards its belief, but does not move it as close to their belief as possible. The following lemma shows that in our model where each agent participates only once, bidding myopically is dominant.\\

\begin{lemma}
Given that agents participate only once, the myopic strategy is dominant. 
\end{lemma}
PROOF. Consider a non-myopic strategy that is dominant for an agent in a market scoring rule based market. The non-myopic strategy must either move the price $\OptionPrice$ towards their belief $\Agent_\AgentBelief$, but less than their budget $\Agent_\AgentBudget$ permits, or away from it. In the case where the non-myopic strategy moves the price away from their belief, the agent will profit based on how much their price movement improves the market price. In order for the agent to make positive profit, the market's price needs to move closer to the underlying true probability. If the agent thinks that this non-myopic strategy is improving the price then they cannot hold belief $\Agent_\AgentBelief$, which is a contradiction. Similarly, if the agent holds belief $\Agent_\AgentBelief$ then it would be strictly preferable to move the price as close as possible to $\Agent_\AgentBelief$ rather than simply closer. This means that the myopic strategy strictly dominates the non-myopic strategy. Since in both cases the myopic strategy strictly dominates therefore the myopic strategy is dominant under this model. $\blacksquare$\\

Therefore, it is safe to assume that all agents will behave myopically.\\

% \subsection{\href{https://www.cs.cmu.edu/~sandholm/liquidity-sensitive automated market maker.teac.pdf}{Practical Liquidity Sensitive Market Maker}}
% The Liquidity Sensitive Market Maker uses the underlying LMSR mechanism but invokes a novel
% function for setting the liquidity parameter $\LMSRb$. The function is:
% $\LMSRb(q) = \LMSRAlpha \sum_{i} q_i$, where $q$ represents the quantity of 
% each option on the market. Binary prediction markets, which
% we have restricted ourselves to, reduce the function to $\LMSRb(q) = \LMSRAlpha (\QuantityYes + \QuantityNo)$. 
% The parameter $\LMSRAlpha$  
% between $(0,1]$ that represents the commission the market maker skims
% off each transaction. A larger $\LMSRAlpha$ yields a higher commission.\\

% The LSMM uses this formula for setting $\LMSRb$ to make the market
% maker profitable and to make it liquidity sensitive, meaning that
% the market maker charges traders differently depending on the market
% depth. \\

% \textbf{Advantages}\\
% \begin{enumerate}
% \item{Path Independence - any way the market moves from one state to another state yields the same payment or cost to the traders in aggregate [Hanson 2003]}
% \item{Liquidity Sensitive - the market adjusts to periods with low or high activity. The market maker decreasingly subsidizes the market as activity rises.}
% \item{Guaranteed Profit - the market maker has unbounded profit but bounded
% loss at near 0.}
% \end{enumerate}

% \textbf{Disadvantages} \\
% \begin{enumerate}
% \item{Translation Variance - all prices sum beyond unity. (No direct mapping to a probability though it does provide a tight range.)}
% \item Market makers are incentivized to raise their commission to 1, which not only hurts traders, but also increases the valid probability range
% and decreases the number of traders who are willing to trade with the market maker. See $\frac{1}{n} - \LMSRAlpha(n - 1)\ln(n) \leq p(q_i) \leq \frac{1}{n} + \alpha\ln(n)$.
% \end{enumerate}

% \subsection{LSMM Algorithms}
% \begin{algorithm}[H]
% \TitleOfAlgo{$\LMSRb(\LMSRAlpha)$}
% \SetKwInOut{Input}{input}
% \SetKwInOut{Output}{output}
% \Input{alpha $\LMSRAlpha$}
% \Output{liquidity parameter $\LMSRb$}
% return $\LMSRAlpha (\Market_\QuantityYes + \Market_\QuantityNo)$\;
% \end{algorithm}

% \subsection{Luke's New MM}

% The current function is $\LMSRb(q) = \alpha \left[\sum_{i}(q_i) + t \right]$ where $t$ represents the number of transactions that have occurred.
