\section{Market Makers}
In these experiments we will test the following three LMSR market markers.

\subsection{\href{http://mason.gmu.edu/~rhanson/mktscore.pdf}{Logarithmic Market Scoring Rule}}
LMSR is a strictly proper scoring rule developed by Robert Hanson. LMSR uses a logarithmic cost function: \\
$C(q_1,q_2) = b\ln(e^{\frac{q_1}{b}} + e^{\frac{q_2}{b}})$ where $q_1$ and $q_2$ represent the number of shares acquired for each of the binary events: 1 and 2. The cost charged to a trader wanting to buy $q_a$ shares on event 1 and $q_b$ shares on event 1 is: $C(q_1 + q_a, q_2 + q_b) - C(q_1,q_2)$. Traders are charged for their movement in the market prediction. $\LMSRb$ is the liquidity parameter set ex ante by the market maker. It controls how much the market maker can lose and also adjusts how easily a trader can change the market price. The market maker always loses up to $\LMSRb\ln(2)$. A large $\LMSRb$ means that it would cost a lot to move the market price while a small $\LMSRb$ makes large swings relatively inexpensive.\\

The instantaneous price of LMSR market $\LMSRMarketMaker$, which is also the market's prediction for the option $\Option$, is quoted with: $p_x = \frac{e^{\frac{q_1}{\LMSRb}}}{e^{\frac{q_1}{\LMSRb}} + e^{\frac{q_2}{\LMSRb}}}$. \\

\textbf{Advantages}\\
\begin{enumerate}
\item{Path Independence - any way the market moves from one state to another state yields the same payment or cost to the traders in aggregate [Hanson 2003]}
\item{Translation Invariance - all prices sum to unity. (Direct mapping to a probability.)}
\end{enumerate}

\textbf{Disadvantages} \\
\begin{enumerate}
\item{Liquidity Insensitive - the market cannot adjust to periods with low or high activity. The market maker must set the liquidity parameter based on their prior belief, but has little to no guidance on how to set it.}
\item{Guaranteed Loss - the market maker cannot profit and has a guaranteed bounded loss.}
\end{enumerate}

\subsection{LMSR Algorithms}
\begin{algorithm}[H]
\SetAlgoLined
\TitleOfAlgo{cost}
\KwData{lmsr market maker $\MarketMaker$, quantity $\Quantity$, direction $\Option_\OptionDirection$}
\KwResult{cost $\Money$}
initialization\;
oldScore = $\MarketMaker\text{.getLiquidityParameter()}\exp(\frac{\MarketMaker\text{.getYesQuantity()}}{\MarketMaker\text{.getLiquidityParameter()}}) + \exp(\frac{\MarketMaker\text{.getNoQuantity()}}{\MarketMaker\text{.getLiquidityParameter()}})$\;
\eIf{direction == YES}{
	newScore = $\MarketMaker\text{.getLiquidityParameter()}\exp(\frac{\MarketMaker\text{.getYesQuantity()} + \Quantity}{\MarketMaker\text{.getLiquidityParameter()}}) + \exp(\frac{\MarketMaker\text{.getNoQuantity()}}{\MarketMaker\text{.getLiquidityParameter()}})$\;
}{
	newScore = $\MarketMaker\text{.getLiquidityParameter()}\exp(\frac{\MarketMaker\text{.getYesQuantity()}}{\MarketMaker\text{.getLiquidityParameter()}}) + \exp(\frac{\MarketMaker\text{.getNoQuantity()} + \Quantity}{\MarketMaker\text{.getLiquidityParameter()}})$\;
}

return newScore - oldScore\;
\end{algorithm}

\begin{algorithm}[H]
\SetAlgoLined
\TitleOfAlgo{priceToYesShares}
\KwData{lmsr market maker $\MarketMaker$}
\KwResult{quantity $\Quantity$}
initialization\;
return $\MarketMaker$.priceToShares(quantity, true)\;
\end{algorithm}

\begin{algorithm}[H]
\SetAlgoLined
\TitleOfAlgo{priceToNoShares}
\KwData{lmsr market maker $\MarketMaker$}
\KwResult{quantity $\Quantity$}
initialization\;
return $\MarketMaker$.priceToShares(quantity, false)\;
\end{algorithm}

\begin{algorithm}[H]
\SetAlgoLined
\TitleOfAlgo{capitalToYesShares}
\KwData{lmsr market maker $\MarketMaker$}
\KwResult{money $\Money$}
initialization\;
return $\MarketMaker$.capitalToShares(money, true)\;
\end{algorithm}

\begin{algorithm}[H]
\SetAlgoLined
\TitleOfAlgo{capitalToNoShares}
\KwData{lmsr market maker $\MarketMaker$}
\KwResult{money $\Money$}
initialization\;
return $\MarketMaker$.capitalToShares(money, false)\;
\end{algorithm}

\begin{algorithm}[H]
\SetAlgoLined
\TitleOfAlgo{priceToShares}
\KwData{lmsr market maker $\MarketMaker$, desiredPrice $\OptionPrice$, direction boolean}
\KwResult{quantity $\Quantity$}
initialization\;
\eIf{direction}{
price = $\Market_\MarketPrice$\;
side = $\MarketMaker$.getYesQuantity()\;
top = $\MarketMaker$.getNoQuantity()\;
}{
price = (1-$\Market_\MarketPrice$)\;
side = $\MarketMaker$.getNoQuantity()\;
top = $\MarketMaker$.getYesQuantity()\;
}
return $\MarketMaker$.getLiquidityParameter()*log((price*exp(top/$\MarketMaker$.getLiquidityParameter()))/(1-price)) - side\;
\end{algorithm}

\begin{algorithm}[H]
\SetAlgoLined
\TitleOfAlgo{capitalToShares}
\KwData{lmsr market maker $\MarketMaker$, money $\Money$, direction boolean}
\KwResult{quantity $\Quantity$}
initialization\;
\eIf{direction}{
side = $\MarketMaker$.getYesQuantity()\;
top = $\MarketMaker$.getNoQuantity()\;
}{
side = $\MarketMaker$.getNoQuantity()\;
top = $\MarketMaker$.getYesQuantity()\;
}
return $\MarketMaker$.getLiquidityParameter() * log(exp(money/ $\MarketMaker$.getLiquidityParameter() + log(pow(e, $\MarketMaker$.getYesQuantity() / $\MarketMaker$.getLiquidityParameter()) + pow(e, $\MarketMaker$.getNoQuantity() / $\MarketMaker$.getLiquidityParameter()))) - exp(top / $\MarketMaker$.getLiquidityParameter())) - side\;
\end{algorithm}

\subsection{\href{https://www.cs.cmu.edu/~sandholm/liquidity-sensitive automated market maker.teac.pdf}{Practical Liquidity Sensitive Market Maker}}
The LSMM uses the underlying LMSR mechanism but invokes a novel
function for setting the liquidity parameter $\LMSRb$. The function is:
$\LMSRb(q) = \alpha \sum_{i} q_i$. $q$ represents the quantity vector
for each option available from the market maker. In a binary
prediction market, $q$ has two values which represent the quantity
outstanding for shares of YES and NO. $\alpha$ is an ex ante parameter 
between $[0,1]$ that represents what commission the market maker takes
off of each transaction. A larger $\alpha$ will result in a higher
profit.\\

The LSMM uses this formula for setting $\LMSRb$ to make the market
maker profitable and to make it liquidity sensitive, meaning that
the market maker charges traders differently depending on the market
depth. \\

\textbf{Advantages}\\
\begin{enumerate}
\item{Path Independence - any way the market moves from one state to another state yields the same payment or cost to the traders in aggregate [Hanson 2003]}
\item{Liquidity Sensitive - the market adjusts to periods with low or high activity. The market maker decreasingly subsidizes the market as activity rises.}
\item{Guaranteed Profit - the market maker has unbounded profit but bounded
loss at near 0.}
\end{enumerate}

\textbf{Disadvantages} \\
\begin{enumerate}
\item{Translation Variance - all prices sum beyond unity. (No direct mapping to a probability though it does provide a tight range.)}
\item Market makers are incentivized to raise their commission to 1, which not only hurts traders, but also increases the valid probability range
and decreases the number of traders who are willing to trade with the market maker. See $\frac{1}{n} - \alpha(n - 1)\ln(n) \leq p(q_i) \leq \frac{1}{n} + \alpha\ln(n)$.
\end{enumerate}

\subsection{LSMM Algorithms}
\begin{algorithm}[H]
\SetAlgoLined
\TitleOfAlgo{getLiquidityParameter}
\KwData{alpha $\alpha$}
initialization\;
return $\alpha (\MarketMaker\text{.getYesQuantity()} + \MarketMaker\text{.getNoQuantity()})$\;
\end{algorithm}

\subsection{Luke's New MM}
\commentl{It needs to be homogenous degree 1.}

It needs to have an understanding of time. LSMM scales with price but does not reward traders that insert information when the market has less information to offer them in exchange. It incentivizes you to wait until the
end to trade. \\

It needs to incentivize lowering the commission in a competition. LSMM 
encourages MMs to ramp up their take in a group setting, which is
suboptimal for the market at large since that disincentivizes trading. \\

The current function is $\LMSRb(q) = \alpha \left[\sum_{i}(q_i) + t \right]$ where $t$ represents the number of transactions that have occurred.
