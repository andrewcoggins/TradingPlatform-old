\section{Benchmarks}
We establish two benchmarks for the market making mechanisms. The first is market maker profit. This is essential because LMSR and our trade incentivizing market makers often operate at no profit or at a loss. The second is accuracy. We will evaluate accuracy according to expected price as determined by our three equilibrium concepts.

\subsection{Profit}
  \begin{definition} (Market Maker Revenue)
  \label{def:mmr}
   Given a market maker $\OptionPrice_\Market$ and a set of participating agents
   $\Agents_\Option$, the revenue obtained from $\OptionPrice_\Market$ is defined as
     $$R(\OptionPrice_\Market, \Agents_\Option) = 
      \sum_{\Agent\in\Agents_\Option} \left[
	\int_{\Time = 0}^{\Time = \Option_\Time} 
	  \Price_\Market(\Option_0,\Agent,\Time, \AgentStrategy^{0}_{\Agent}(\Time))\AgentStrategy^{0}_{\Agent}(\Time)dt
	  + \int_{\Time = 0}^{\Time = \Option_\Time} 
	  \Price_\Market(\Option_1,\Agent,\Time, \AgentStrategy^{1}_{\Agent}(\Time))\AgentStrategy^{1}_{\Agent}(\Time)dt\right]$$
  \end{definition}
  
    \begin{definition} (Market Maker Cost)
  \label{def:mmc}
   Given a market maker $\OptionPrice_\Market$ and a set of participating agents
   $\Agents_\Option$, the cost to $\OptionPrice_\Market$ is defined as
     $$ C(\OptionPrice_\Market, \Agents_\Option) =
     \sum_{\Agent\in\Agents_\Option} \left[
	\int_{\Time = 0}^{\Time = \Option_\Time} 
	\Revealer(\Option_\Event)\AgentStrategy^{0}_{\Agent}(\Time)dt
	+\int_{\Time = 0}^{\Time = \Option_\Time} 
	\Revealer(\Option_\Event)\AgentStrategy^{1}_{\Agent}(\Time)dt
	\right]$$
  \end{definition}

      \begin{definition} (Market Maker Profit)
  \label{def:mmf}
   Given a market maker $\OptionPrice_\Market$ and a set of participating agents
   $\Agents_\Option$, the profit of $\OptionPrice_\Market$ is defined as
     $$ P(\OptionPrice_\Market, \Agents_\Option) = R(\OptionPrice_\Market, \Agents_\Option) 
     - C(\OptionPrice_\Market, \Agents_\Option)$$
  \end{definition}
  
\begin{definition} (Profit-Maximizing Market Maker).
\label{def:pmmm}
Among all Market Makers $\MarketMakers$, given a set of participating agents $\Agents_\Option$ find the one that maximizes Profit:
$$ PM(\MarketMakers, \Agents_\Option) = arg max_{\Agent \in \Agents_\Option} P(\OptionPrice_\Market, \Agents_\Option)$$
\end{definition}

\subsection{Accuracy}
We will use the following definitions of accuracy from our equilibrium concepts.
<<<<<<< HEAD
\mydef{Rational Expectations}, \mydef{Prior Information}, and \mydef{Prior Information Timing} provide three definitions for ground truth against which we can compare the final prediction for our \mydef{market makers}. \\
=======

\mydef{Rational Expectations} and \mydef{Prior Information} provide three definitions for ground truth against which we can compare the final prediction for our \mydef{market makers}. All three equilibria suggest that the market's final prediction will predict the true outcome. \\

\mydef{Rational Expectations} implies that the final price will be the average of each agent's signal since all agents have equal budgets. As we have restricted signals to the set of ${0,1}$, the expected price will simply be the number of 1 signals divided by the total number of agents $\frac{|\Agents_1|}{|\Agents|}$. Given that the aggregate signal is truthful, this implies that the market is 100\% accurate. For example, with a signal vector $\left[0,1,1,0,1\right]$, the expected price is .6 and the prediction will be correct. \\

\mydef{Prior Information} implies that the final price will be between the weighted average and .5 because it is biased towards the initial price. The prediction will still be accurate just with lower confidence. For example, with a signal vector $\left[0,1,1\right]$, the first agent will target .25 since it is the Bayesian update between their prior 0 and the price .5. The second agent will target .5 since it is the Bayesian update between their prior 1 and the price .25 giving .25 double weight. The third agent will target .625 since it is the Bayesian update between their prior 1 and the price .5 giving .5 triple weight. The outcome will still be correct at .625, but less than .66 which is the RE prediction in this case.\\

% \mydef{Prior Information Timing} implies something that I will calculate.\\
>>>>>>> newInterfaces
