\section{Motivation}
The Logarithmic Market Scoring Rule (LMSR), while popular in the theoretical
literature, faces several critical barriers to adoption. The first is that
the \mydef{market maker} is bounded loss, which is a non-starter for most
real world applications. The second is that the \mydef{market maker} is not
sensitive to liquidity. This means that a improperly configured \mydef{market maker}
might allow any agent with a positive budget to drastically swing the price or at
the other extreme, not allow any agent to modify the price much at all. LMSR is
well studied, however, because it has Path Independence and Translation Invariance.
Many other theoretical \mydef{market makers} have been proposed that also have
these two properties, but with additional benefits, and some offer tradeoffs in
order to gain more desirable properties. This work seeks to quantify these
tradeoffs empirically.

\section{Goals}
To test five different market maker mechanisms for their liquidity sensitivity,
profit expectation, and accuracy.  


\section{Definitions}

We denote time by $\Time\in\mathbb{R}_+$. \\%We denote money by $\Money\in\mathbb{R}$.\\

An \mydef{event} has an outcome where we restrict our attention to a binary outcome. 
The outcome is equal to YES in case the event occurs and NO otherwise. 
We denote the YES outcome with a 1 and the NO outcome with a 0.
We assume there is a way to unambiguously determine the outcome of an event. \\

$\Revealer$ is an oracle for mapping an \mydef{event} to an outcome.
We denote the outcome of event $\Event$ under $\Revealer$ as $\Revealer(\Event) \in \{0,1\}$.\\

An \mydef{option} $\Option$ is a security that yields a return depending on the outcome of an 
event $\Option_\Event$. Each \mydef{option} has a direction $\Option_\OptionDirection \in \{0, 1\}$
and a strike time $\Option_\Time$ when $\Revealer(\Option_\Event)$ will be evaluated. 
The \mydef{option} will convert to \$1 if $\Revealer(\Option_\Event)$ equals 
$\Option_\OptionDirection$, otherwise it converts to \$0. Two options are said to be
complementary if they trade opposite directions in the same event. Given an
event $\Event$, we denote the complementary options as 
$\Option_{0}$ and $\Option_{1}$.
\\

Given an \mydef{option} $\Option$,
an \mydef{agent} $\Agent\in\Agents_\Option$ has a \mydef{private belief} 
$\Agent_\AgentBelief \in [0,1]$ 
and a \mydef{budget} $\Agent_\AgentBudget \in \mathbb{R}_+$.
The agent's private belief $\Agent_\AgentBelief$ 
is the subjective probability that the agent
assigns to the outcome of event $\Option_\Event$ being $\Option_\OptionDirection$
at strike time.\\

A \mydef{prediction market} $\Market$ trades complementary options. 
Formally, a prediction market is a tuple $\Market = \left<\Option_0,\Option_1, 
\Agents_{\Option_0} \cup \Agents_{\Option_1} \right>$ where each \mydef{agent} 
$\Agent\in\Agents_\Market = \Agents_{\Option_0} \cup \Agents_{\Option_1}$ purchases 
either some number of $\Option_0$ or $\Option_1$ options paying the price quoted by the 
\mydef{market maker}.\\


A \mydef{market maker} $\OptionPrice_\Market$ is a function 
that maps an option $\Option$, an agent $\Agent \in \Agents_\Option$ 
and a quantity $\Quantity\in\mathbb{R}_+$ at time $\Time \in [0,\Time_\Option)$ 
to a price $\OptionPrice(\Option,\Agent,\Quantity, \Time) \in \mathbb{R}_+$. 
By definition $\OptionPrice(\Option,\Agent, \Time) = \Revealer(\Option_\Event)$ if 
$\Time \ge \Time_\Option$.

%The \mydef{market depth} $\MarketDepth$ is the number of \mydef{agents} that have 
%participated in the prediction market, i.e. $|\Agents_\Option|$.


\section{Market Makers}
In these experiments we will test the following five market markers.
\subsection{Logarithmic Market Scoring Rule}
\subsection{Luke's Online Budget Weighted Average}
\subsection{Yiling and Jen's Expert Weighted Majority}
\subsection{Luke's Weighted Majority}
\subsection{Practical Liquidity Sensitive Market Maker}

\section{Setup}

\subsection{Types}
A \mydef{BasicAgent} follows from the definition of these \mydef{market makers} as truthful
mechanisms. Each \mydef{BasicAgent} reports its belief $\AgentBelief$ and budget $\AgentBudget$
truthfully when requested by the mechanism. Each \mydef{BasicAgent} is rational and profit
maximizing.

\subsection{Metrics}
This research will use the following metrics to evaluate each of the five mechanisms.
\subsubsection{Liquidity Sensitivity}
This research uses a novel definition of the vague concept of liquidity sensitivity. The need
for a unified definiton stems from the known issue that LMSR can provide \"too little\" or 
\"too much\" weight to each agent depending on its liquidity parameter and market depth. One extreme
example us that an infinitesimally low liquidity parameter will allow an \mydef{agent} with a nonzero
budget to dictate the market price regardless of how many \mydef{agents} have already participated in
the market. Our novel definition attempts to quanity consistency across \mydef{market depths} without
overly constraining implementations. \\

We define a \mydef{market maker} as \mydef{liquidity sensitive} if two \mydef{agents} that have identical
transactions in the \mydef{market} will have equal impacts on the \mydef{market price} proportional to 
the \mydef{market depth} $\MarketDepth_{1}$ at time $\Time_{1}$ of the first purchase, the \mydef{market depth} 
$\MarketDepth_{2}$ at time $\Time_{2}$ of the second purchase, and the difference in time $\Time_{1} - \Time_{2}$.\\

We formalize \mydef{liquidity sensitive} as the function $\Delta = f(\MarketDepth_{1},\MarketDepth_{2},\Time_{\Delta})$
having a constant $\Delta$.\\

\subsubsection{Market Maker Profit}
We define the \mydef{market maker profit} on prediction market $\Market$ at time $\Time$ as\\
$\sum_{\Agent\in\Agents_\Option} [
\int_{\Time} 
\Price(\Option_0,\Agent,\Time, \AgentStrategy^{0}_{\Agent}(\Time))\AgentStrategy^{0}_{\Agent}(\Time)dt
+\int_{\Time} 
\Price(\Option_1,\Agent,\Time, \AgentStrategy^{1}_{\Agent}(\Time))\AgentStrategy^{1}_{\Agent}(\Time)dt
-\int_{\Time} 
\Revealer(\Option_\Event)\AgentStrategy^{0}_{\Agent}(\Time)dt
-\int_{\Time}
\Revealer(\Option_\Event)\AgentStrategy^{1}_{\Agent}(\Time)dt]$

\subsubsection{Social Welfare}
We define the profit of an agent as the sum of the realized value of its \mydef{options} minus the
sum of the cost of each of its \mydef{options}.

We define \mydef{social welfare} as the \mydef{market maker profit} plus the sum of each agent's profit.

\subsubsection{Accuracy}
Since the accuracy of a prediction market depends on the accuracy of each of its participants
as well as a usually unknown ground truth value, we utilize several well studied definitions
to evaluate each mechanism's accuracy.
\textbf{Regret}
We define the \mydef{regret} of a mechanism as the difference between the mechanism's accuracy
and the accuracy of the most accurate agent $\Agent\in\Agents$, who can be thought of as the
best expert. The most accurate agent $\Agent\in\Agents$ is the agent $\Agent$ whose belief is
most frequently aligned with the outcome over n trials.

\textbf{Expectation}
We define the \mydef{expectation accuracy} as the difference between the expert's prediction
and the weighted expected value for each agent $\Agent\in\Agent$. Each agent's belief is weighted
based off their budget and then an expected value is calculated.

\textbf{Mean Squared Error}
We define the \mydef{Mean Squared Error} as $\frac{1}{n}\sum_{t=1}^{n} \big(\Market_{0,1} - O_{0,1}\big)^2$. MSE
is computed across n agents $\Agent\in\Agents$. 

\subsubsection{Precision}
\mydef{Precision} is the consistency of the \mydef{market maker}. \mydef{Precision} is computed by taking a \mydef{market maker}
and a set of agents $\Agents$, and randomizing the order that the agents enter the market. \mydef{Precision} is the
percentage of predictions across all simulations that are the same as the majority outcome. This means that if 6 of
the 10 simulations have the \mydef{market} predicting $Outcome_{1}$ then the \mydef{Precision} is 0.6.

\section{Experimental Design}
The following simulations will be run by having an identical set of agents enter each \mydef{market} using each
of the five \mydef{market makers}. After each simulation, the \mydef{market makers} will be scored for every
metric. After all the simulations, the \mydef{market makers} will be ranked for each metric depending on the emperical
results. A simulation will be run for the cross product of the following characteristics:
\begin{enumerate}
  \item Agent Number: 3, 5, 10, 50, 100, 1000, 10000.
  \item Agent Belief: Uniform $[0,1]$, Normal w/ Mean 0.5, Beta(0.5,0.5).
  \item Agent Budget: Uniform $[0,100]$, Uniform $[0,10]$, Normal w/ Mean 5, 10*Beta(0.5,0.5).
\end{enumerate}
