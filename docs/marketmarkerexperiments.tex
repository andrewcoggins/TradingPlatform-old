\section{Motivation}

So... LMSR... while great in theory (why?? path independence), sucks in practice (why??).
We want to 

\section{Goals}
To test five different market maker mechanisms for their liquidity sensitivity,
profit expectation and accuracy.  


\section{Definitions}

We denote time by $\Time\in\mathbb{R}_+$. We denote money by $\Money\in\mathbb{R}$.\\

An \mydef{event} has an outcome where we restrict our attention to a binary outcome. 
The outcome is equal to YES in case the event occurs and NO otherwise. We assume there 
is a way to unambiguously determine the outcome of an event. The mechanism for making
the determination is $\R$.\\

$\R$ is an oracle for mapping an \mydef{event} to an outcome.\\

An \mydef{option} is a security that yields a return depending on the outcome of an 
event. Each \mydef{option} has a price, a $\Option_{\Time}$ when $\R(\Option)$ will be
evaluated, and an outcome. The \mydef{option} will convert to \$1 if $\Option_{outcome}$
equals $\R(\Option)$ otherwise it converts to \$0.\\

The function $\OptionPrice(\Option,\Time)$ at time $\Time$ reflects the likelihood 
of the $\Option_{outcome}$ being realized at the strike date $\Option_{\Time}$.\\

An \mydef{agent} $\Agent\in\Agents$ has a belief $\AgentBelief_{\Agent\Time} \in [0,1]$ 
at time $\Time$. The $\AgentBelief_{\Agent\Time}$ reflects the agent's private belief
in the expected likelihood of the $\Option_{outcome}$ being realized. This belief maps
directly to the price that the agent is willing to pay for the $\Option$.\\

A \mydef{prediction market} $\Market$ trades outcomes in an event. Formally, a prediction market 
is a tuple $\left<\Option_0,\Option_1, \Agents, \Book\right>$. Each \mydef{agent} 
$\Agent\in\Agents$ purchases some number of $\Option_0$ and $\Option_1$ paying the price 
quoted by the \mydef{prediction market} at each time $\Time$ when the \mydef{agent} $\Agent$ 
made the purchases. A prediction market has a \mydef{book} $\Book$ that maps $\Agent\in\Agents$ to a tuple $\left<\mathbb{R},\mathbb{R}\right>$ which tracks the number of $\Option_0$ and $\Option_1$ 
purchased by that \mydef{agent} $\Agent$.

A \mydef{book} $\Book_{\Market,\Time}$ accepts an \mydef{agent} $\Agent$ and returns the number of 
each \mydef{option} $\Option$ that \mydef{agent} $\Agent$ has purchased at time $\Time$. A 
\mydef{book} $\Book$ at time $\Time$ records all the transactions made up to time $\Time$.\\

A \mydef{market maker} $\MarketMaker$ is \mydef{predciction market} and strategy for setting prices
on each \mydef{option} $\Option_0$ and $\Option_1$ at time $\Time$. Each \mydef{market maker} has
their own function $\Price(\Option,\Time,\Agent,\mathbb{R})$ where the quantity $\mathbb{R}$ is the
number of the option desired by the \mydef{agent} $\Agent$, and the output is the money $\Money$
required by the \mydef{market maker} $\MarketMaker$ from the \mydef{agent} $\Agent$. \\

We formalize the pricing fnction as$\Price:\left<\Option,\Time,\Agent,\mathbb{R}\right> \rightarrow \Money$.\\

A \mydef{market} is defined as a \mydef{prediction market}.\\

\section{Market Makers}
In these experiments we will test the following five market markers.
\subsection{Logarithmic Market Scoring Rule}
\subsection{Luke's Online Budget Weighted Average}
\subsection{Yiling and Jen's Expert Weighted Majority}
\subsection{Luke's Weighted Majority}
\subsection{Practical Liquidity Sensitive Market Maker}

\section{Setup}

\subsection{Types}

\subsection{Metrics}
\subsubsection{Liquidity Sensitivity}
\subsubsection{Market Maker Profit}
\subsubsection{Social Welfare}
\subsubsection{Accuracy}
\paragraph{Regret}
\paragraph{Expectation}
\paragraph{Mean Squared Error}
\subsubsection{Precision}